\documentclass{book}
\usepackage{makeidx}
\usepackage[utf8]{inputenc}

 \title{Modular Application in Rails to combine Games and SCORM Resources}
 \author{Alberto Benito Campo\\Escuela de Ingenieros de Telecomunicaciones\\Universidad Politécnica de Madrid}
 \date{June 2016}

 
 
\begin{document}
\begin{titlepage}
\maketitle
\end{titlepage}

\chapter{Introduction}
\paragraph{This bachellor degree thesis is driven by the possibility to explore new paths to motivate students during their education. During my early years, I have realized how video games have influenced me and how they were one of the main time expenditures. Also, one of my main knowledge sources. I never understood, why it was not applied to education itself(in general), or the way was applied, it wasn't very accessible to teachers.}

\paragraph{In this project I am giving a solution to this problem, considering technical requirements of a software project. In particular taking into account, modularity, future improvements, code re-utilization}
\chapter{Modular Rails}

\paragraph{Assuming we all know benefits and contraries of creating modular application, I am proposing a way about what is needed to create an modular application in Rails.}
\paragraph{
	In Rails 3-4, there is a programmatic mechanism called Engine, this mechanism let you to create, as Ruby On Rails webpage says, miniature applications that provide functionality to host applications. Rails Application can be considered as a `supercharged´ engine because it inherits a lot of its behaviour from a Rails Engine.
}
\paragraph{For this project I have followed a guide proposed by (enter name of author) to create a modular application in Ruby On Rails. He proposes to create an empty application where you mount all funcionality as engines, begining with a core which will be the engine to support users and then other functionalities in other engines. For this project I have considered the following structure:}

\paragraph{
	-Core: contains functionality of Users and Images.
	-Scorm Creator: contains games and maps game events with scorm files.
	-Scorm System: manages incoming and outconmig Scorm Files (course files) in the application.
	-Mini Editor(a fork of Vish Editor, as an independant entity): tool in charge of creation the courses that will be included in games. 
}



\chapter{SCORM and Games}


\chapter{Course Editor}

\chapter{References}
\paragraph{guides.rubyonrails.org/engines.html}
\end{document}